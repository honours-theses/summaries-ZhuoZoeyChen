\documentclass[12pt]{article}
\usepackage[utf8]{inputenc}

\title{Background Knowledge Notes}
\author{Zhuo Chen}

\begin{document}

\maketitle

This document includes notes on the relevant background knowledge in the field of $\lambda$-calculus.
This document is written in order to assist Zhuo's PhD project.

\newpage 

\begin{enumerate}
    \item \textbf{Evaluation Context}: (E, E[·]) is a $\lambda$-term or a metaexpression 
representing a family of $\lambda$-terms with a special variable [·] called the 
hole. Every evaluation context E[·] represents a \textit{context} rule:
$$\frac{e \rightarrow e'}{E[e]\rightarrow E[e']}$$
    \item \textbf{Contextual Equivalence}: two phrases of programming language are 
contexually equivalent (=ctx) if any occurences of the first phrase in a 
complete program can be replaced by the second phrase without affecting the 
observable results of excuting the program.
    \item \textbf{Operational Semantics}: a category of formal programming language 
semantics in which certain desired properties of a program, such as correctness, safety 
or security, are verified by constructing proofs from logical statements about its execution 
and procedures, rather than by attaching mathematical meanings to its terms. (denotational 
semantics does it by attaching mathematical meanings to its terms)
    \item \textbf{$\beta$-reduction} revisit: 
        $$ \frac{ }{(\lambda v.M)N \rightarrow_{\beta} M[v:=N]} $$
        $$ \frac {M \rightarrow_{\beta} M'}{MN \rightarrow_{\beta}M'N}$$
        $$ \frac {N \rightarrow_{\beta} N'}{MN \rightarrow_{\beta}MN'}$$
        $$ \frac {M \rightarrow_{\beta} M'}{(\lambda v.M) \rightarrow_{\beta}(\lambda v.M')}$$ ($\xi$-rule)
    \item \textbf{Weak $\lambda$-calculus}:
        \begin{itemize}
            \item $\lambda$-terms: same as (strong) $\lambda$-calculus
            \item $\beta$-reduction: substitution only 
            \item Does not have $\xi$-rule 
            $$ \frac {M \rightarrow_{\beta} M'}{(\lambda v.M) \rightarrow_{\beta}(\lambda v.M')}$$
            \item reduction does not go under abstraction 
            \item not confluent 
        \end{itemize}
    \item \textbf{Weak Strategy}
    \item \textbf{Weak Reduction}
    \item \textbf{Normalisation Property}: 
    a rewrite system has the (strong) normalization property or is terminating if every term is strongly normalizing; 
    that is, if every sequence of rewrites eventually terminates with an irreducible term, also called a normal form. 
    A rewrite system may also have the weak normalization property, meaning that for every term, there exists at least 
    one particular sequence of rewrites that eventually yields a normal form, i.e., an irreducible term.
    \item \textbf{Sharing}: multiple use of a same expression, which then needs to 
    be evaluated only once, improving efficiency.
    \item \textbf{Standardisation Theorem}: It states that if a term M $\beta$-reduces to 
    a term N, then there exists a standard $\beta$-reduction sequence from M to N.

    The standardisation theorem states that given a derivation $d:t\rightarrow_{\beta}^* s$
    it is always possible to rearranging it into a standard derivation $e:t\rightarrow_{\beta}^* s$
    that computes the same result but selecting redexes from left to right, this is the 
    completeness property of standard derivation.
    \item \textbf{Explicit Substitution} (ES): An intermediate formalism that - by 
    de-composing the higher-order substitution operation into more atomic steps - 
    allows a better understanding of the execution of complex languages.
    \item \textbf{Cut-Elimination} revisit:
        \begin{itemize}
            \item cut: Having $\Gamma \models A, \Delta$ and $\Pi, A \models \Lambda$,
            can conclude $\Gamma, \Pi \models \Delta, \Lambda$.
            \item cut-elimination: $\Gamma \models A$ and $\Pi, A \models B$,
            can conclude $\Gamma, \Pi \models  B$.
        \end{itemize}
    \item \textbf{Meta-terms}:terms with metavariables used to represent incomplete programs and proofs.
    Metaterms are terms containing metavariables denoting incomplete programs/proofs in a higher-order unification framework 
    \item \textbf{Decent}:
     a term t is said to be decent in the calculus $\lambda Z$ if every subterm $v$ appearing in some substituted subterm $u[x/v]$ of $t$ is $\lambda Z$ strongly normalising. 
     As an example, the term $x[x/(y y)][y/\lambda w.w w]$ is decent in $\lambda es$ since $y$ $y$ and $\lambda w.w w $are both $\lambda es$-strongly normalising, but its Comp2-reduct $x[x/(y y)[y/\lambda w.w w]]$ is not. 
    \item \textbf{Abstraction}: ($\lambda$x. m); function definition (M is a lambda term). The variable x becomes bound in the expression.
    \item \textbf{Eager functional language}: In computer programming, eager evaluation, also known as strict evaluation or greedy evaluation, is the evaluation strategy used by most traditional programming languages. 
    In eager evaluation, an expression is evaluated as soon as it is bound to a variable. 
    An opposite alternative to eager evaluation is lazy evaluation, where expressions are evaluated only when a dependent expression is evaluated depending upon a defined evaluation strategy.
       \item \textbf{Linear types}: corresponds to linear logic and ensures that objects are used exactly once, allowing the system to safely deallocate an object after its use.
    \item \textbf{(Strong) invariance Thesis}: all reasonable models of computation also agree, to some extent, in their notions of complexity: ‘Reasonable’
    machines can simulate each other within a polynomially bounded overhead
    in time and a constant-factor overhead in space.”
    \item \textbf{Sublinear time complexity}: An algorithm is said to run in sub-linear time (often spelled sublinear time) if T(n) = o(n). Note that 
    it is little-o here. 
    \item \textbf{Universal $\lambda$-term}: ? 
    \item \textbf{Natural Measures for time and space}: ?
    \item \textbf{Call-by-push-value} (CBPV): is an idealised calculus for functional and imperative programming, introduced as a subsuming paradigm for both call-by-value (CBV) and call-by-name
    (CBN). CBV and CBN can be simulated in CBPV and the translations preserve operational and denotational semantics
    In programming language theory, the call-by-push-value (CBPV) paradigm,[1] inspired by monads, allows writing semantics for lambda-calculus without writing two variants to deal with the difference between call-by-name and call-by-value. To do so, CBPV introduces a term language that distinguishes computations and values, according to the slogan a value is, a computation does; this term language has a single evaluation order. However, to evaluate a lambda-calculus term according to either the call-by-name (CBN) or call-by-value (CBV) reduction strategy, one can translate the term to CBPV using a call-by-name or call-by-value translation strategy, which give rise to different terms. Evaluating the result of the call-by-value translation corresponds to evaluating the original term with the call-by-value strategy; evaluating the result of the call-by-name translation corresponds instead to evaluating the original term with the call-by-name strategy.
    \item \textbf{Constructive Type Theory}:Constructive type theory, (also known as intuitionistic type theory, or Martin-Löf type theory) is a type theory and an alternative foundation of mathematics.
    Intuitionistic type theory has 3 finite types, which are then composed using 5 different type constructors. Unlike set theories, type theories are not built on top of a logic like Frege's. So, each feature of the type theory does double duty as a feature of both math and logic.
    If you are unfamiliar with type theory and know set theory, a quick summary is: Types contain terms just like sets contain elements. Terms belong to one and only one type. Terms like 2+2 and 2·2 compute ("reduce") down to canonical terms like 4. More details: https://plato.stanford.edu/entries/type-theory-intuitionistic/
    \item \textbf{Multi-tape Turing machine}: A multi-tape Turing machine is a variant of the Turing machine that utilizes several tapes. Each tape has its own head for reading and writing. Initially, the input appears on tape 1, and the others start out blank.
    This model intuitively seems much more powerful than the single-tape model, but any multi-tape machine—no matter how many tapes—can be simulated by a single-tape machine using only quadratically more computation time.
    A k-tape Turing machine can be described as a 6-tuple $M = \langle Q, \Gamma, s, b, F, \delta \rangle$ where:
        \begin{itemize}
            \item $Q$ is a finite set of states
            \item $\Gamma$ is a finite set of the tape alphabet 
            \item $s\in Q$ is the initial state 
            \item $b\in \Gamma$ is the blank symbol 
            \item $F\subseteq Q$ is the set of final or accepting states 
            \item $\delta : Q \times \Gamma^k \rightarrow Q \times (\Gamma \times \{L, R, S\})^k$ is a partial 
            function called the transition function, where k is the number of tapes, L is left shift, R is right shift and 
            S is no shift. 
        \end{itemize}
    \item \textbf{Lift}: In category theory, a branch of mathematics, given a morphism f: X → Y and a morphism g: Z → Y, a lift or lifting of f to Z is a morphism h: X → Z such that f = g∘h. We say that f factors through h.
    \item \textbf{Brouwer–Heyting–Kolmogorov interpretation (realizability interpretation)}:
    The interpretation states what is intended to be a proof of a given formula. This is specified by induction on the structure of that formula:
    \begin{itemize}
        \item   A proof of $P \wedge Q$ is a pair $\langle a, b \rangle$ where $a$ is a proof of $P$ and $b$ is a proof of $Q$.
        \item 	A proof of $P \vee Q$ is a pair $\langle a, b \rangle$ where $a$ is 0 and $b$ is a proof of $P$, or $a$ is 1 and $b$ is a proof of $Q$.
        \item	A proof of $P \rightarrow Q$ is a function $f$ that converts a proof of $P$ into a proof of $Q$.
        \item	A proof of $\exists x \in S: \varphi (x)$is a pair $\langle a, b \rangle$ where $a$ is an element of $S$, and $b$ is a proof of $\varphi(a)$.
        \item	A proof of $\forall x \in S: \varphi (x)$ is a function $f$ that converts an element $a$ of $S$ into a proof of $\varphi(a)$.
        \item	The formula $\neg P$ is defined as $P \rightarrow \bot$, so a proof of it is a function $f$ that converts a proof of $P$ into a proof of $\bot$.
        \item	There is no proof of $\bot$ (the absurdity, or bottom type (nontermination in some programming languages)).
    \end{itemize}

The interpretation of a primitive proposition is supposed to be known from context. In the context of arithmetic, a proof of the formula $s = t$ is a computation reducing the two terms to the same numeral.
Kolmogorov followed the same lines but phrased his interpretation in terms of problems and solutions. To assert a formula is to claim to know a solution to the problem represented by that formula. 
For instance $P \rightarrow Q$ is the problem of reducing $Q$ to $P$;
to solve it requires a method to solve problem $Q$ given a solution to problem $P$.
    \item \textbf{Lint/Linter}: is a static code analysis tool used to flag programming errors, bugs, stylistic errors and suspicious constructs. The term originates from a Unix utility that examined C language source code.
    \item \textbf{Linear Types}: corresponds to linear logic and ensures that objects are used exactly once, allowing the system to safely deallocate an object after its use.
    \item \textbf{Linear Logic}: TODO
    \item \textbf{Hoare Logic}: TODO
    \item \textbf{Separation Logic}: TODO
    \item \textbf{Semantics}: TODO
    \item \textbf{Cost Models}: a mathematical model that tries to measure of the cost of executing a program. 
	- Denotational cost models 
		assigns a cost directly to syntax 
	- Operational cost 
		first define a program-evaluating abstract machine. Then determine the time cost by counting the number of steps taken by the abstract machine. 
\end{enumerate}


\end{document}