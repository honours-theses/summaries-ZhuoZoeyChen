\documentclass[12pt]{article}
\linespread{1.25}
\usepackage[utf8]{inputenc}
\usepackage{natbib}
\usepackage{bibentry}
\usepackage{sectsty}
\usepackage[autostyle]{csquotes} 
\nobibliography*


\title{A Verified Cost Model for $\lambda$-Calculus}
\author{Zhuo Chen (z5155600)\\
[0.5cm]University of New South Wales\\ 
School of Computer Science and Engineering}

\begin{document}

\maketitle
\thispagestyle{empty}

\begin{center}
{\small Supervisors: Christine Rizkallah, 
Johannes Åman Pohjola, and Carroll Morgan} \\ 
{\small GSOE 9400 (Engineering Postgraduate Research Essentials)}
\end{center}

\newpage 

\pagenumbering{roman} 

\begin{abstract}
        Developing an effective cost model for $\lambda$-calculus has always been an important topic 
        in the research of computer science as $\lambda$-calculus is one of the fundamental models of 
        computation and provides a foundation for both computability and functional programming. 

        There has been ongoing research in developing cost models for $\lambda$-calculus with different evaluation strategies
        and this project aims to extend this line of work by providing a new effective 
        verified cost model for call-by-value $\lambda$-calculus using HOL.
\end{abstract}

\newpage 

\tableofcontents

\newpage 

\pagenumbering{arabic} 

\section{Introduction}

The $\lambda$-calculus~\cite{DBLP:series/hhl/CardoneH09} is a fundamental model of computation that represents
 functions using applications and abstractions over variables. 
It provides a foundation for computability and functional programming. 
Unlike programs written in an imperative style, functional programs do not specify
 low-level details describing \emph{how} they should run on a computer. Instead, 
  functional languages provide a convenient way of describing \emph{what} functions 
  should compute \emph{to} in a concise mathematical manner. This abstract representation is ideal for proving 
  that a program is correct against its specification. 
  This answers the following important question: Does my program correctly compute the result that it is supposed to?
  Other than functional correctness, another property that is of high interest to us computer scientists 
  is run-time computational complexity. 
  Computational complexity addresses the vital questions: How fast does my program produce an output?
   
  For algorithms, there are classic cost models such as big-O 
  notation that significantly ease the process of analysing computational complexity.
   Creating effective cost models for functional programming languages 
   is a significant challenge and there has been a large body of research
    dedicated to solving this task over the years.
   
  Functional programming languages are associated with evaluation strategies 
  that describe how programs written in these languages evaluate function-calls, specifying the order
   in which these function-calls are evaluated. 
   Examples of such strategies include call-by-value, 
   the strategy used by the ML programming language~\cite{DBLP:books/daglib/0069232}, 
   and call-by-need, 
   the strategy used by Haskell~\cite{DBLP:conf/hopl/HudakHJW07}. These evaluation strategies affect the 
   run-time efficiency of the functional programs. 
    
Over the years, there has been ongoing research~\cite{DBLP:journals/entcs/Accattoli18, DBLP:conf/popl/MoranS99,DBLP:journals/pacmpl/HackettH19} 
in developing cost models for $\lambda$-calculus with different evaluation strategies.
This work provides a solid theoretical basis for creating a useful cost model 
for functional programming. 

This project aims to extend this line of work by providing a new effective 
verified cost model for call-by-value $\lambda$-calculus. 
The verification will be done in the interactive theorem prover (ITP), HOL~\cite{DBLP:conf/tphol/SlindN08},
a software tool which assists human-computer collaborated proofs. 

\newpage 

\section{Literature Review}
    \subsection{Interactive Theorem Prover}

Traditionally, computer scientists carry out the computability proofs including $\lambda$-calculus 
related definitions and proofs in a pen-and-paper style. 
However, pen-and-paper proofs tend to be error-prone, especially when the proof gets longer and more complicated.
Software tools known as theorem provers then comes in handy. 
They provide proof environments which allows machines to check and validate the 
given proofs (or even generate the proofs in some cases).
There are two kinds of theorem provers: interactive and automated. 
Automated theorem provers are able to carry out proofs for given theorems without any human input, however,
they are not capable of proving relatively sophisticated theorems that require long chains of reasoning involving the 
statement of many intermediate lemmas.
Interactive theorem provers are also known as proof assistants, named after how they 
work: carrying out proofs by human-computer collaboration. 
More specifically, interactive theorem provers allow users to encode their mathematical/logical goals
and then to prove them, while having their steps checked by the machine. 
In the case of our project with $\lambda$-calculus, interactive theorem provers 
allow us to describe the syntax and semantics of $\lambda$-calculus with specific evaluation strategies,
and to define and verify the cost model we develop for this specific $\lambda$-calculus. 
There exists many different interactive theorem provers, such as Coq~\citep{coq} 
, Isabelle/HOL~\citep{DBLP:books/sp/NipkowPW02, DBLP:books/sp/NipkowK14} and 
the one we are proposing to use in our project, HOL~\citep{DBLP:conf/tphol/SlindN08}. 

HOL is a theorem prover developed for interactive theorem proving in higher order logics
(in particular, predicate calculus with terms from simple type theory). 
HOL is interfaced to Standard ML (Standard Meta Language,~\cite{DBLP:books/daglib/0069232}), a general-purpose functional programming language 
which is popular in the development of theorem provers. 
This interface allows users to denote the terms and theories of the logic,
express and apply the proof strategies, and develop logical theories in HOL.

The general approach to work with HOL is that we can first create an empty new theory,
then optionally import existing theories (equivalent to libraries in other languages), 
and then we can make new definitions and/or define and prove new theorems under this theory.
After we are done with this theory, we will be able to export it and use it as a 
library when we are making other new theories. 

    \subsection{Cost models for $\lambda$-Calculus}
\paragraph{The $\lambda$-calculus} is a model of computation that expresses computation in a mathematical manner 
by the use of function abstraction and application over variables using variable binding and substitution.
The strategies we use to reduce terms within $\lambda$-calculus are called evaluation strategies.
Different evaluation strategies will result in different evaluation steps, which then leads to different costs.
As a result, they will need different cost models. 

\paragraph{Cost models for $\lambda$-Calculus} Over the years, there has been a lot of work on creating formal cost models for 
	computation~\cite{DBLP:journals/entcs/Accattoli18, DBLP:journals/pacmpl/HackettH19, 
	DBLP:journals/igpl/BucciarelliKV17, DBLP:conf/popl/MoranS99, 
	DBLP:journals/mscs/Carvalho18}. 
    
    (In)Efficiency and Reasonable Cost Models~\citep{DBLP:journals/entcs/Accattoli18} by Accattoli lays a solid groundwork about cost models which clears the common confusion 
    about what it means for an evaluation strategy to be \emph{reasonable}, \emph{efficient} and \emph{standard}.
    Accattoli points out that the key features for reasonable strategies are termination and sub-term property.
    He then represents the back-and-forth simulations between $\lambda$-calculus and Turing machines after 
    introducing cost models for Turing machines and $\lambda$-calculus and their properties. 
    Between the two directions of simulations, 
    simulating $\lambda$-calculus using Turing machines is more complicated due to the size explosion resulting from $\beta$-reduction.
    After this, Accattoli introduces useful sharing and discusses about the optimisation strategies to improve the efficiency of 
    reasonable cost models. The strategies includes optimal sequential evaluation, optimal parallel evaluation, sharing of sub-terms and sharing 
    of computations and reasonable sharing of computations. 
    In the end, the relationship between being standard and reasonable is discussed through explanation about 
    why leftmost-outermost strategy is reasonable.
    
    The models from the other papers mentioned earlier have different trade-offs between the accuracy of the bounds 
	they provide and the difficulty of using the model. Formalising these models will provide 
	a deeper understanding of their trade-offs and help in identifying their limitations.
	This will provide an insight for creating further models that combine the benefit of 
	the existing ones. Such benefits include tighter upper and lower bounds for 
	the estimated cost as well as simplifying the use of the model itself.

    \subsection{Formally verified cost models}
	In the recent years, there have been exciting developments in the use of 
	interactive theorem provers for proving computational complexity of programs, 
	such as intersection types verified in Agda ~\cite{DBLP:journals/igpl/BucciarelliKV17}
	and verified programming of Turing machines in Coq~\cite{DBLP:conf/cpp/0002KW20}.

    The verified programming of Turing machines in Coq presents a framework for the verified programming of multi-tape Turing machines in Coq.
    The implementation includes four layers of abstractions, named as L0 - L3 from low to high. The highest abstraction L3 allows a user to 
    implement nontrivial algorithms as Turing machines and verify their correctness, as well as time and space complexity compositionally. 
    The project also includes case studies, where they verify a translation from multi-tape to single-tape machines as well as a universal Turing machine, both 
    with polynomial time overhead and constant factor space overhead.

    As one would guess, there are also ones about formalisation of Lambda calculus.
    One of our interest is Formal Small-Step Verification of a Call-by-Value Lambda Calculus
    Machine~\cite{DBLP:conf/aplas/KunzeS018}, which provides a formal verification of 
    an abstract machine for a call-by-value $\lambda$-calculus with de Bruijn terms, simple substitution, and small-step semantics.
    
    With this groundwork, The Weak Call-by-value Lambda-calculus is reasonable for both time and space~\cite{DBLP:journals/pacmpl/ForsterKR20} 
    by Forster, Kunze and Roth is then published.
    They prove that the weak call-by-value $\lambda$-calculus (\textbf{L}) is a reasonable machine with respect to the natural 
    time and space measures and enables the formal verification of complexity-theoretic proofs concerning complexity classes, both on paper and in proof assistants. 
    They prove the reasonability of weak call-by-value $\lambda$-calculus by verifying the both ways simulation relation between \textbf{L} and Turing machines. 
    The complicated direction, which is simulating \textbf{L} using Turing machines, is defined and verified by interleaving two simulation 
    strategies. These two simulation strategies are called substitution-based strategy and heap-based strategy respectively. The former simulates a 
    reduction sequence naively as given by the reduction rules of \textbf{L}. 
    The latter uses closures and keep track of the values assigned to variables in an environment instead of 
    executing any substitution if a $\beta$-reduction is simulated and thus allows for structure sharing. 
    In more detail, two abstract machines implementing these strategies 
    are introduced in order to analyse these strategies on a more semantic level. 
Their work is exciting and brings potential future work including: 
\begin{enumerate}
    \item Extending to sublinear time or space classes;
    \item Covering more $\lambda$-calculi with different evaluation strategies;
    \item Mechanising basic complexity theory;
    \item Investigating whether the result ($P \subseteq PSPACE$) can be proved without reference to Turing machines, 
    for instance by implementing the space-aware interleaving simulation in a universal $\lambda$-term;
    \item Investigating whether these exists a higher-order interpretation of space complexity that does not exhibit (exponential) size explosion. 
\end{enumerate}


    Building on top of this work,
    A Mechanised Proof of the Time Invariance Thesis for the Weak Call-By-Value
               $\lambda$-Calculus~\cite{DBLP:conf/itp/0002KSW21} 
    is published by Forster, Kunze, Smolka, and Wuttke in 2021.
    In this paper, they achieve the following:
    \blockquote{\textit{We have presented the first mechanised proof of an instance of the time invariance thesis, 
    connecting L with Turing machines with a polynomial overhead in time. We prove two variants, respectively concerned 
    with simulation of one model of computation on the other, and with the computability of relations on boolean strings.}}
    This verified equivalence (in time) between L and Turing machine
    gives us the flexibility of choosing between L and Turing machines when we need to prove the time 
    invariance thesis for a different model of computation, depending on which one is more suitable. 
    Although this is an extension of The Weak Call-by-value Lambda-calculus is reasonable for both time and space~\cite{DBLP:journals/pacmpl/ForsterKR20}, it does not close all the gaps. 
    Most of the future work from the previous paprer still remains open; on top of that, this paper 
    only provides the proof for the time invariance thesis for \textbf{L} and doesn't cover space complexity, 
    which is another potential future direction. 
    The framework provided by this paper can also be simplified further by developing an automatic generation 
    of correctness proofs using meta-programming tools for Coq, however, we will not consider this as part of 
    potential future work we would do as it will not be as creative and we are using HOL instead of Coq.

\newpage

\section{Methodology}

As mentioned earlier, pen-and-paper proofs tend to be error-prone. 
Thus in order to provide and verify an effective cost model for Lambda calculus, 
we will need to use interactive theorem provers. That been said, starting from 
theorem prover directly is not the most efficient way as the mechanisation itself 
is often complicated and hard for human to organise thoughts in there directly with the code. 
Thus we will first come up with a pen-and-paper proof as it is more abstract and easier 
to understand for human. We will then encode this proof in the interactive theorem prover 
of choice (HOL) and show that it is in fact machine checkable and thus correct. 

The experiment objects will include $\lambda$-calculus with different evaluation strategies:
weak call-by-value, call-by-value and call-by-push-value~\cite{DBLP:conf/tlca/Levy99}. 
We will use Turing machine as the intermediate model of computation to verify our cost model.
On top of that, we can use verified compiler to ensure that the complexity is preserved during 
compilation. 

When we manage to obtain the verified cost model, we are able to use it to formally verify 
functional programs and evaluate its effectiveness. 

\newpage 

\section{Research Plan}
The aim of this research is to develop a formally verified effective cost model for $\lambda$-calculus.
We propose the following research plan. 
Note that subsection~\ref{4.0} talks about the relevant research work 
we have done so far by the point of the write-up of this proposal. 
From subsection~\ref{4.1} onwards, we then introduce the detailed research plan 
we have for the future. 

Potential publication in first year could include a survey paper and/or a 
small extension on subsection~\ref{4.0}. From subsection~\ref{4.1} onwards,
each subsection could result in half to one publication.

\setcounter{subsection}{-1}

\subsection{Porting Existing Formal Verification}
\label{4.0}
At this stage, we port the Coq formalisation of a Call-by-value 
$\lambda$-calculus with small-step semantics~\cite{DBLP:conf/aplas/KunzeS018}
into HOL. 

We first make sure that we fully understanding the paper and the published 
Coq formalisation by reading them multiple times and discussing them
during the project meetings. In order to look into the detailed 
proof procedure, we first learn basic syntax, tactics and theorems of Coq by going through Coq documents 
and libraries. We then install the required version of Coq on our machine 
and step through the proofs to ensure that we understand how the 
pen-and-paper proof is translated into Coq. 

After all these preparation, we carry out the translation
from the given Coq code into HOL manually. 

This exercise helps us to better understand the published work 
and enables us to reuse the relevant proofs in HOL if needed. 

\subsection{Development of a Theoretical Model}
\label{4.1}
	We plan to start by creating a pen-and-paper generalisation of a 
	recently developed effective cost model for weak call-by-value 
	$\lambda$-calculus~\cite{DBLP:journals/pacmpl/ForsterKR20}.
	Our generalisation will extend the current model to the full call-by-value $\lambda$-calculus
    and a call-by-push-value $\lambda$-calculus. 
    We will compare the two models to see which one will be more suitable:
    effective and can be interpreted into HOL without too much hassle. 
    In the case we can't decide between the two, we go to the next stage 
    with both models and figure out which one is better along 
    the way of formalisation. 

\subsection{Formal Verification of the Model.}
	
The verification tool we propose to use here is HOL, 
so that we can extend and further develop existing work in HOL.
 In more details, there exist models for $\lambda$-calculus and Turing machines 
 as well as partial proofs of their equivalence. I have worked on this formalisation 
 project in the past, which will ease completing this task.

This stage includes three essential steps in order to formally 
verify the cost model developed by pen-and-paper.
First, we will extend the existing $\lambda$-calculus and Turing machines 
models that are formalised in HOL. Then, we will extend and verify 
the \emph{simulation relation} from $\lambda$-calculus to Turing machines in HOL. 
This verification ensures that programs written in
$\lambda$-calculus can be converted into Turing machines while preserving 
the complexity.
Finally, we will verify our cost model by comparing it to the existing cost 
models of Turing machines through our simulation relation.  
 
At that stage, we will be ready to apply our cost model to verify programs written 
in $\lambda$-calculus.
   
\subsection{Formal Verification of Efficient Compilers.}
	
	The functional programming language CakeML~\cite{CakeML} is accompanied 
	by a verified compiler. We plan to use our cost model to extend the 
	guarantees provided by CakeML's verified compiler to also ensure 
	that the complexity of programs is preserved during compilation.

    This will provide a formal proof that the CakeML compiler maintains
	the computational complexity of programs. 
    Together with the verification of our cost model in HOL, we can then 
    say that we have obtained a fully verified cost model from the level 
    of specification down to the level of compiled code. 

\subsection{Application of the Model.}
	
	At this stage, we will have a cost model for a functional programming language as
	well as formal proofs that this model's complexity is maintained all 
	the way down to Turing machines. The final step of this project will be 
	to write functional programs and formally verify their computational complexity. 


\newpage 

\section{Conclusion}

$\lambda$-calculus is not only a fundamental model of computation but also 
an important mathematical formal system. It is an essential tool for 
both computer scientists and mathematicians, so having a verified effective
cost model for it has the following significance:
\begin{enumerate}
	\item Formally verifying the correctness of programs is important, however,  
	it is not always sufficient. For many applications, it is vital to also verify 
	that they fall in a certain complexity class. One example of such applications 
	is cryptographic protocols for which the correctness argument fundamentally 
	relies on the hardness of decrypting information. Another such example is the 
	assumption that compilers do not affect the asymptotic complexity of the programs 
	they compile. For instance, programmers often implicitly assume that if they write 
	a program that runs in linear time, the compiled machine 
	code will also run in linear time. Verifying compiler correctness is therefore not 
	sufficient. One also needs to verify that it does not change the asymptotic complexity 
	of the programs it compiles.
	
	Being able to compute the complexity of execution strategies for 
	$\lambda$-calculus provides the necessary infrastructure for formally reasoning about
	the complexity of programs on an abstract level. 
	This provides higher assurance in the complexity models used.
	\item Cost models developed on pen and paper are not machine checked thus creating the possibility of error,
	 whilst a verified cost model would be guaranteed to be correct. 
	There has been some recent work on using 
	interactive theorem provers for estimating the computational complexity of programs.
	This is an active area of research with many open problems, such as 
	creating tighter estimates of complexity which would provide higher assurance in the 
	areas described above.
\end{enumerate}

This project aims to extend the work~\cite{DBLP:journals/pacmpl/ForsterKR20} and \cite{DBLP:conf/itp/0002KSW21} 
by providing a new effective verified cost model for call-by-value $\lambda$-calculus in HOL.
This will contribute significantly to the development of cost models for functional programming languages 
as call-by-value is one of the most common evaluation strategies. This also opens gate to developing 
a verified cost model for call-by-push-value $\lambda$-calculus.

\newpage 

\bibliographystyle{plain}
\bibliography{refs}

\end{document}