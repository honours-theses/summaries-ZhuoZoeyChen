\documentclass[12pt]{article}
\usepackage[utf8]{inputenc}
\usepackage{natbib}

\title{Literature Review Summaries}
\author{Zhuo Chen}

\begin{document}

\maketitle

This document includes short summaries for all papers in the related field of $\lambda$-calculus and its cost models. 
This document is written in order to assist Zhuo's PhD project.

\newpage 

\section{(In)Efficiency and reasonable cost models~\cite{DBLP:journals/entcs/Accattoli18}}
\paragraph{Author:} Beniamino Accattoli 
\paragraph{Summary:} This paper is about time cost models for the $\lambda$-calculus. It discusses the relationship of the evaluation 
strategies being reasonable and being efficient as well as them being standard and being reasonable.

This paper first discusses the relationship between being reasonable and efficient and emphasises that they are two 
different properties. It then talks about cost models for Turing machines and $\lambda$-calculus and their properties. 
It points out that the key features for reasonable strategies are termination and sub-term property, 

This paper then go into dept about how to encode a reasonable model (e.g. Turing machine) in $\lambda$-calculus and 
how to encode $\lambda$-calculus in a reasonable model (e.g. Turing machine/RAM). The latter is the more complicated one 
as it needs to spend a lot of time dealing with size explosion.

Next, this paper talks about useful sharing, including its definition, importance, where it can be used and difficulty of using it. 

Following that, this paper discusses optimisation strategies to improve the efficiency of 
reasonable cost models, strategies including optimal sequential evaluation, optimal parallel evaluation, sharing of sub-terms and sharing 
of computations and reasonable sharing of computations. 

In the end, the paper talks about the relationship between standard and reasonable through its explanation about 
why leftmost-outermost strategy is reasonable.

\paragraph{Keywords:} Lambda calculus, cost models, sharing, computational complexity, functional programming.

\section{The theory of calculi with explicit substitutions revisited~\cite{DBLP:conf/csl/Kesner07}}
\paragraph{Author:} Delia Kesner
\paragraph{Summary:} This paper first talks about the survey they have done about work in the domain of calculi with explicit substitution, 
then establish a general theory of explicit substitutions for lambda-calculus that enjoys fundamental properties including confluence on 
metaterms (and thus on terms), simulation of one-step $\beta$-reduction, strong normalisation of typed terms, preservation of $\beta$-strong normalisation, 
simulation of one-step $\beta$-reduction and full composition. 

In the survey part, it talks about the issues that calculi with explicit substitution are suffering from 
, mentions Mellies' counter-example, proposes three possible solutions and their respective issues. It then
addresses calculi with explicit substitution, the properties that they satisfy and the ones that they don't.

The paper then introduces $\lambda$es-calculus, which leads to the second part of the paper. 
The paper first introduces $\lambda$es-calculus's syntax, including its definition, equation C and reductions rules.
It then shows how to prove that $\lambda$es-calculus is confluent on meta-terms and 
how to prove that it preserves $\beta$-strong normalisation. The latter is proved by introducing another two 
helper calculus $\lambda$esw-calculus and $\Gamma_I$-calculus. In the end it introduces The typed $\lambda$es-calculus.

\paragraph{Keywords:} Lambda calculus, explicit substitutions, survey, $\lambda$es-calculus.

\bibliographystyle{plain}
\bibliography{refs}

\end{document}